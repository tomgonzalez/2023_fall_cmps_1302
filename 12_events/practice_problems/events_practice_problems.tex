\documentclass[11pt]{exam}
\usepackage{minted}
\usepackage{hyperref}
\newcommand{\mil}[1]{\mintinline{java}{#1}}

\pagestyle{headandfoot}
\runningheadrule
\firstpageheader{Java}{JavaFX Events Practice Problems}{}
\runningheader{Java}
{Practice Problems, Page \thepage\ of \numpages}
{\today}
\firstpagefooter{}{}{}
\runningfooter{}{}{}

\newbox\allanswers
\setbox\allanswers=\vbox{}

\newenvironment{answer}
{%
  \global\setbox\allanswers=\vbox\bgroup
  \unvbox\allanswers
}%
{%
  \bigbreak
  \egroup
}

\newcommand{\showallanswers}{\par\unvbox\allanswers}
	
\begin{document}

\begin{questions}

\question Write a JavaFX application that displays a Label containing the opening sentence or two from your favorite book.  When the user clicks a button, display the title of the book that contains the quote in a second label.
        
\question    Write a JavaFX application that contains a Button.  Disable the Button after the user clicks it.  Save the program as FXFrameDisableButton2a.java.

Use myButton.setDisable(true);
    


\question Write a JavaFX application that contains a Button.  Disable the Button after the user has clicked it at least eight times.  At that point, display a Label that indicates "That's enough!"  Save the program as FXFrameDisableButton2b.java.


    
\question Create a JavaFX application with at least six labels that contain facts about your favorite topic.  Every time the user clicks a Button, remove one of the labels and add a different one.  Save the program as FXFacts.java. 

\question A dog kennel boards dogs at a cost of 50 cents per pound per day.  Write a JavaFX class that allows the user to enter a dog's weight and number of days to be boarded and then displays the total price for boarding.

\question Write a JavaFX application that allows the user to input a value in miles, converts the value to feet, and then displays the results to the user.  Display explanatory text with the values.  For example, 8.5 miles is equal to 44880.0 feet.

\question  Write a JavaFX program that allows the user to enter a number of minutes and then converts it both to hours and days.  For example, 6000 minutes equals 100 hours and equal 4.167 days.
\end{questions}	

\end{document}