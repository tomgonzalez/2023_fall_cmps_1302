\documentclass{beamer}

\usepackage[english]{babel}
\usepackage{minted}
\newcommand{\mil}[1]{\mintinline{java}{#1}}
\usepackage{upquote}
\usepackage{graphicx}
\usepackage{tikz}
\usetikzlibrary{matrix,backgrounds}
\usepackage{hyperref}

\title[Events]{JavaFX}
\subtitle{Events} % (optional)

\author[]{Dalton State College} 

%\institute[Dalton State College]

\date[T. Gonzalez]{T. Gonzalez}

% If you wish to uncover everything in a step-wise fashion, uncomment
% the following command: 

%\beamerdefaultoverlayspecification{<+->}

\begin{document}

\begin{frame}

	\titlepage
	
\end{frame}

\begin{frame}

    \frametitle{Event Driven Programming vs. Procedural Programming}
    
    Event-driven programming is a programming paradigm in which the flow of the program is determined by event such as user actions (mouse clicks, key presses), sensor outputs, or messages from other programs or threads.
    
    \bigskip
        
    Event-driven programming is the dominant paradigm used in graphical user interfaces and web applications.        
    
\end{frame}

\begin{frame}
    
    \frametitle{Events}
    
    An \textbf{event} is an action or occurrence recognized by a program, often originating from the external environment, such as the user.
    
    \bigskip
    
    When an event occurs, such as the mouse click or button key press, an object is generated by the system.
    
    \bigskip
    
    The program must detect the generation of the event and respond to it.

    \bigskip
        
    Every GUI program has an \textbf{event loop} which watches for the generation of events.
    
    \bigskip
    
    The process of responding to an event is called \textbf{event handling}.
    
    \bigskip
    
    
\end{frame}



\begin{frame}[fragile]

	\frametitle{\mil{Button} and \mil{ActionEvent}}
	
    When a user clicks on a \mil{Button}, an \mil{ActionEvent} is generated.
    
    \bigskip
    
    To handle the \mil{ActionEvent}, call the \mil{setOnAction()} in the \mil{Button} object. 
    
    \bigskip
    
    \mil{button.setOnAction(e -> method())}
    
    \bigskip
    
    The expression \mil{e -> method()} is called a lambda expression.
    
    \bigskip
    
    The method on the right side of the lambda expression can be a built-in method or one that you provide.
    
    \bigskip
    
    See ButtonActionEventFirstExample.java
	
\end{frame}

\begin{frame}

    \frametitle{In-Class Problem}
    
Write a JavaFX program with two buttons and a \mil{Text} object.  Clicking on one button should move the \mil{Text} object to the left and clicking on the other button should move the \mil{Text} object to the right.

\bigskip

To build the GUI:

\bigskip

Start with a BorderPane.\\
Insert an HBox in the bottom of the BorderPane.\\
Insert two Buttons in the HBox.\\
Insert a Pane in the center of the BorderPane.\\
Insert a Text object in the Pane.    
\end{frame}

\begin{frame}[fragile]

	\frametitle{\mil{TextField} and \mil{ActionEvent}}
	
    When presses the Enter key while in a \mil{TextField}, an \mil{ActionEvent} is generated.
    
    \bigskip
    
    To handle the \mil{ActionEvent}, call the \mil{setOnAction()} in the \mil{TextField} object. 
    
    \bigskip
    
    \mil{textField.setOnAction(e -> method())}
    
    
    \bigskip
    
    See TextFieldActionEventFirstExample.java
	
\end{frame}

\begin{frame}

    \frametitle{In-Class Problem}
    
Write a JavaFX program that converts miles to kilometers and vice versa.  The GUI should have two \mil{TextField}s, one for miles and one for kilometers.  If you enter a value in the miles \mil{TextField} and press the enter key, the corresponding kilometer value is displayed in the kilometers \mil{TextField}.  Likewise, if you enter a value in the kilometer \mil{TextField} and press the Enter key, the corresponding mile value is displayed in the mile \mil{TextField}.    
\end{frame}

\begin{frame}

    \frametitle{In-Class Problem}
    
 Write a JavaFX program that lets the user enter a loan amount and loan period in number of years and displays the monthly and total payments for each interest rate starting from 5\% to 8\%, with an increment of 1/8.  The user interface should include two \mil{TextField}s, one for the loan amount, one for the number of years.  The user interface should also contain a \mil{Button}.  When the user clicks the button, the results should be displayed in a \mil{TextArea}.  The next slide shows a sample run.
\end{frame}

\begin{frame}[fragile]
\begin{verbatim}
Loan Amount:  10000
Number of Years:  5

Interest Rate      Monthly Payment     Total Payment
 5.000%             $188.71             $11,322.74
 5.125%             $189.28             $11,357.13
 5.250%             $189.85             $11,391.59
 ...
 7.875%             $202.17             $12,129.97
 8.000%             $202.76             $12,165.83 
\end{verbatim}
\end{frame}
\end{document}
\begin{frame}
    \frametitle{In-Class Problem}
    
    Write a JavaFX application that displays a Label containing the opening sentence or two from your favorite book.  When the user clicks a button, display the title of the book that contains the quote in a second label.
    
\end{frame}

\begin{frame}
    \frametitle{In-Class Problem}
    
    Write a JavaFX application that contains a Buton.  Disable the Button after the user clicks it.  Save the program as FXFrameDisableButton2a.java.

Use myButton.setDisable(true);
    
\end{frame}

\begin{frame}

    \frametitle{In-Class Problem}
    
Write a JavaFX application that contains a Button.  Disable the Button after the user has clicked it at least eight times.  At that point, display a Label that indicates "That's enough!"  Save the program as FXFrameDisableButton2b.java.
\end{frame}

\begin{frame}

    \frametitle{In-Class Problem}
    
Create a JavaFX application with at least six labels that contain facts about your favorite topic.  Every time the user clicks a Button, remove one of the labels and add a different one.  Save the program as FXFacts.java.    
\end{frame}

\end{document}