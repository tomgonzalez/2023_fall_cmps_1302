\documentclass{beamer}

\usepackage[english]{babel}
\usepackage{minted}
\newcommand{\mil}[1]{\mintinline{java}{#1}}
\usepackage{upquote}
\usepackage{graphicx}
\usepackage{tikz}
\usetikzlibrary{matrix,backgrounds}
\usepackage{hyperref}

\title[Introduction to JavaFX]{JavaFX}
\subtitle{Introduction to JavaFX} % (optional)

\author[]{Dalton State College} 

%\institute[Dalton State College]

\date[T. Gonzalez]{T. Gonzalez}

% If you wish to uncover everything in a step-wise fashion, uncomment
% the following command: 

%\beamerdefaultoverlayspecification{<+->}


\begin{document}

\begin{frame}

	\titlepage
	
\end{frame}

\begin{frame}
    
    \frametitle{Graphical User Interfaces}
    
    A graphical user interface (GUI) is a form of user interface that allows users to interact with a device through graphical icons and visual indicators.
    
    \bigskip
    
    Most users expect to interact with computers using a GUI.
    
    \bigskip
    
    GUI programs are event-driven.  User actions such as a clicking a mouse button or pressing a key on the keyboard generate events, and the program responds to the events as they occur. 
\end{frame}
\begin{frame}[fragile]

	\frametitle{JavaFX}
	
	JavaFX is a software platform for creating GUI desktop applications and internet applications. 

    \bigskip

    JavaFX is intended to replace the older Java GUI platform, Swing.
    
    \bigskip
    
    We will be using JavaFX 8 in this class.  

\end{frame}
	
\begin{frame}[fragile]

	\frametitle{First JavaFX Example}
	
    See HelloWorldFX.java
	
\end{frame}

\begin{frame}[fragile]

    Line 1:  \mil{import javafx.application.Application;}
    
    \bigskip
    
    The JavaFX classes we will write will be child classes of the \mil{Application} class.
    
    \bigskip
    
    The \mil{Application} class is the entry point for JavaFX applications.
    
    \bigskip
    
    Whenever a JavaFX application is launched, the JavaFX runtime does the following in order:
    
    \begin{itemize}
        \item Constructs an instance of the specified \mil{Application} class
        \item Calls the \mil{init()} method.
        \item Calls the \mil{start()} method.
        \item Waits for the application to finish.
        \item Call the \mil{stop()} method.
    \end{itemize}
    
    \bigskip
    
    The \mil{start()} method is abstract and must be overridden.  We will use this method to do the setup of our application.
    
\end{frame}

\begin{frame}[fragile]

    Line 2:  \mil{import javafx.stage.Stage;}
    
    \bigskip
    
    A \mil{Stage} object represents a window on the computer screen. 
    
\end{frame}

\begin{frame}[fragile]

    Line 3:    \mil{import javafx.scene.Scene;}
      
    \bigskip
    
     A \mil{Scene} object can be filled with GUI components such as buttons, menus, check boxes, and many more.
    
    \bigskip
    
     A \mil{Stage} object is used to display a \mil{Scene} object.
     
\end{frame}

\begin{frame}[fragile]

    Line 4:  \mil{import javafx.scene.layout.Pane;}    
       
    \bigskip
           
    \mil{Pane} is an example of a GUI component known as a container.  A container can contain other GUI components including other containers.
    
    \bigskip
 
     A \textbf{container} is any descendant of the \mil{javafx.scene.Parent} class.
     
     \bigskip
     
     In our applications, \mil{Scene} objects will have at least one container that contains all other components in the \mil{Scene}.  This component will be referred to as the \textbf{root component}.
     
     \bigskip     
     
     There are many different containers that can be used as the root component of a \mil{Scene}.  This first example just happens to use \mil{Pane}.
     
\end{frame}

\begin{frame}[fragile]

    Line 7:  \mil{public class HelloWorldFX extends Application}
    
    \bigskip
    
    The \mil{HelloWorldFX} class is a child class of the \mil{Application} class.
    
    \bigskip
    
    Since the \mil{start()} method of the \mil{Application} class is \mil{abstract}, our class must provide an implementation.
    
\end{frame}

\begin{frame}

    Line 10:  \mil{public void start(Stage stage)}
    
    \bigskip
    
    We will use the \mil{start()} method to initialize applications.
    
    \bigskip
    
    The \mil{start()} method has one parameter which is a \mil{Stage} object.  
    
    \bigskip
    
    The \mil{Stage} object passed to the \mil{start()} method is constructed by the system and represents the main window of the program.
         
\end{frame}

\begin{frame}[fragile]

    Line 15:  \mil{Pane root = new Pane();}

    \bigskip
    
    This creates the \mil{Pane} object that be the root component of the \mil{Scene} that we will build.
    
    \bigskip
    
    The first example is very simple and will not contain any other GUI components, but in subsequent examples, we will start placing other GUI components into the root component.
    
\end{frame}

\begin{frame}[fragile]

    Line 19:  \mil{Scene scene = new Scene(root, 400, 250);}
    
    \bigskip
    
    Create a \mil{Scene} object to display on the \mil{Stage}.
    
    \bigskip
    
    The \mil{Scene} object needs a root component.
    
    \bigskip
    
    The numbers 400 and 250 set the width and the the height of the \mil{Scene}, respectively.  These arguments are optional.

\end{frame}

\begin{frame}[fragile]

    Line 22:  \mil{stage.setScene(scene);}
    
    \bigskip

    Tells the \mil{stage} object that it will be displaying the \mil{scene} object.    

\end{frame}

\begin{frame}[fragile]

    Line 25:  \mil{stage.setTitle("Hello World!");}
    
    \bigskip
    
    Sets the title of the \mil{stage} object.
    
\end{frame}

\begin{frame}[fragile]

    Line 25:  \mil{stage.setTitle("Hello World!");}
    
    \bigskip
    
    Sets the title of the \mil{stage} object.

\end{frame}

\begin{frame}[fragile]

    Line 28:  \mil{stage.show();}
    
    \bigskip
    
    Tells the \mil{stage} object to show itself.
    
\end{frame}

\begin{frame}[fragile]
    
    Line 41:  \mil{Application.launch(args);}
    
    \bigskip
    
    Calls the \mil{static} method \mil{launch()} in the \mil{Application} class to launch the application and passes the value of the array \mil{args} to the application.
    
    \bigskip
    
    In turn, this will eventually trigger the \mil{start()} method.

\end{frame}



\end{document}