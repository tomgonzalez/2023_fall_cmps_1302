\documentclass{beamer}

\usepackage[english]{babel}
\usepackage{minted}
\newcommand{\mil}[1]{\mintinline{java}{#1}}
\usepackage{upquote}
\usepackage{graphicx}
\usepackage{tikz}
\usetikzlibrary{matrix,backgrounds}
\usepackage{hyperref}

\title[TextFields]{JavaFX}
\subtitle{TextFields} % (optional)

\author[]{Dalton State College} 

%\institute[Dalton State College]

\date[T. Gonzalez]{T. Gonzalez}

% If you wish to uncover everything in a step-wise fashion, uncomment
% the following command: 

%\beamerdefaultoverlayspecification{<+->}


\begin{document}

\begin{frame}

	\titlepage
	
\end{frame}

\begin{frame}
    
    \frametitle{\mil{TextField}}
    
    A \mil{TextField} object allows a user to enter a single line of unformatted text.  

\end{frame}
	
\begin{frame}[fragile]

	\frametitle{\mil{TextField} Constructors}
	
    \mil{TextField()}  Creates an empty \mil{TextField}.
    
    \bigskip
    
    \mil{TextField(String text)} Creates a \mil{TextField} with initial text.
	
\end{frame}

\begin{frame}[fragile]

    \frametitle{Accessing and Changing a TextField Object's Text}
    
    Use the \mil{getText()} method to access a \mil{TextField}'s text.
    
    \bigskip
    
    Use the \mil{setText()} method to change a \mil{TextField}'s text. 
    
\end{frame}

\begin{frame}

    See  TextFieldExample.java.
    
\end{frame}

\begin{frame}

    \frametitle{In-Class Exercise}
    
    Create a JavaFX application with a \mil{Pane} root component containing two \mil{Label}s, two \mil{TextField}s, and a \mil{Button}.
    
\end{frame}
\end{document}