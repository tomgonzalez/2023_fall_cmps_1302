\documentclass{beamer}

\usepackage[english]{babel}
\usepackage{minted}
\newcommand{\mil}[1]{\mintinline{java}{#1}}
\usepackage{upquote}
\usepackage{graphicx}
\usepackage{tikz}
\usetikzlibrary{matrix,backgrounds}
\usepackage{hyperref}

\title[Buttons]{JavaFX}
\subtitle{Buttons} % (optional)

\author[]{Dalton State College} 

%\institute[Dalton State College]

\date[T. Gonzalez]{T. Gonzalez}

% If you wish to uncover everything in a step-wise fashion, uncomment
% the following command: 

%\beamerdefaultoverlayspecification{<+->}


\begin{document}

\begin{frame}

	\titlepage
	
\end{frame}

\begin{frame}
    
    \frametitle{\mil{Button}}
    
    A \mil{Button} object provides a simple button control.
    
    \bigskip
   
    Buttons can be pressed by the user to trigger events in your application.
    
    \bigskip
    
    Buttons can contain text and images.  

\end{frame}
	
\begin{frame}[fragile]

	\frametitle{Button Constructors}
	
    \mil{Button()}  Creates an empty button.
    
    \bigskip
    
    \mil{Button(String text)} Creates a button with specified text as label.
    
    \bigskip
    
    \mil{Button(String text, Node graphic)}  Creates a button with supplied text and graphic for its label.
	
\end{frame}

\begin{frame}[fragile]

    \frametitle{Purpose of Buttons}
    
    The primary purpose of a \mil{Button} is to produce an action when it is clicked.
    
    \bigskip
    
    When a button is clicked an \mil{ActionEvent} object is generated.
    
    \bigskip
        
    Your application can watch for generated \mil{ActionEvent}s and then implement \mil{EventHandler}s to process the \mil{ActionEvent}.
    
    \bigskip
    
    This type of programming is called \textbf{event-driven programming} and is the primary programming paradigm in GUI programming. 
    
    \bigskip
    
    We will talk about how to handle events later.
\end{frame}

\begin{frame}

    See  ButtonExample.java.
    
\end{frame}
\end{document}